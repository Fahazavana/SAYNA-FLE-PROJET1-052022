\documentclass[12pt]{article}
\usepackage[T1]{fontenc}
\usepackage[utf8]{inputenc}
\usepackage[french]{babel}

\author{Jean Lucien RANDRIANANTENANA}
\title{Vers un français
impéccable (B1)\\SAYNA-FLE-PROJET1-052022}
\begin{document}
\maketitle

\section{Les temps du passés}
\subsection{Je complète le texte en conjuguant le verbe entre parenthèse à l'imparfait ou au passé composé}


Le mois dernier, je (se rendre) à la foire aux vins de Strasbourg. Ce (être) un grand événement
alors je (préférer) prendre les transports en commun pour y aller. Il y (avoir) une foule
innombrable. Ce salon (représenter) le moyen de développer mon réseau puisque je
(souhaiter) faire carrière dans une maison vinicole.
Pour valider mon diplôme, je (devoir) trouver un stage. Je (aller) visiter différents stands et je
(rencontrer) de nombreux professionnels. Je (avoir) un bon contact avec un des responsables.
Il me (poser) beaucoup de questions et à la fin de l’entretien, il (vouloir) que je lui donne mon
CV. Une semaine plus tard, il me (appeler) pour passer un entretien et il me (embaucher).


\subsection{Ajouter deux phrases au plus-que-parfait dans le texte précédent}
\begin{enumerate}
	\item
	\item
\end{enumerate}

\subsection{Compléter le texte en conjuguant le verbe à l’imparfait ou au passé composé}
Ce (être) un jour de février. Il (faire) une température glaciale et il (pleuvoir) des cordes.
Je (revenir) du travail quand je (rencontrer) une ancienne collègue. Elle me (ne pas voir) immédiatement.
Je la (appeler), je (se présenter) et nous (discuter) à l’abri dans la rue. 
Elle me (expliquer) qu’elle (être) divorcée et qu’elle (avoir) des enfants. Nous (se souvenir) des anecdotes du travail quand je (réaliser) qu’il (être) déjà tard.
Ma femme me (attendre) pour le repas. Alors, je (prendre) son numéro de téléphone et nous (se saluer).
 \subsection{Ajouter quatre phrases au plus-que-parfait dans le texte précédent}
 \begin{enumerate}
	\item
	\item
	\item
	\item
 \end{enumerate}
 

 \section{Indicatif ou subjonctif}


\subsection{Conjuguer le verbe entre parenthèses au subjonctif présent.}
\begin{enumerate}
	\item J’ai bien peur qu’elle ne (pouvoir) pas venir à mon anniversaire.
	\item Mon père est si content que je (faire) des études.
	\item Nous économisons pour que notre projet de tour du monde (se réaliser).
	\item De crainte que les enfants ne les (découvrir), les parents cachent les cadeaux.
	\item J’aimerais beaucoup que tu (venir) avec moi voir cette exposition.
	\item Il est important que tu (savoir) la vérité.
	\item Je ne pense pas qu’elle (être) de très bonne humeur aujourd’hui.
	\item Ils sont ravis que leur fils (avoir) un travail.
	\item Ils voudraient que nous y (aller) avec eux.
	\item Je suis tellement contente que tu (vouloir) bien m’aider.
\end{enumerate}
\subsection{Choisir la forme correcte du verbe.}
\begin{enumerate}
	\item J’espère vraiment que tu obtiennes / vas obtenir ce travail.
	\item Nous aimerions partir / que nous partions en vacances.
	\item Elle a fait beaucoup d’effort pour que nous passons / passions une bonne soirée.
	\item Je crois qu’il ne puisse / pourra pas venir.
	\item Je ne trouve pas que cette robe lui aille / va bien.
	\item C’est le livre que tu connais / connaisses.
	\item Je ne veux pas que tu es / sois déçue.
	\item Ils ont eu peur que notre train part / parte avec du retard.
	\item Je doute que le chauffeur de taxi sache / sait comment aller chez elle.
	\item Il est certain que ses notes sont / soient à la hauteur de son investissement personnel.
\end{enumerate}
	\subsection{Répondre aux questions suivantes}

\begin{enumerate}
	\item Tu crois qu’ils sont déçus ? Non, je ne crois pas 
	\item Vous pensez qu’elle a assez d’argent ? Non, nous ne pensons pas
	\item Tu trouves que ce plat est bon ? Non, je ne trouve pas 
	\item Tu ne crois pas qu’elle soit malade ? Si, je crois 
	\item Vous trouvez qu’il fait froid ? Oui, je trouve . . .
\end{enumerate}
	\subsection{transformation de texte}
Ma maison est petite et un peu sombre.


Les voisins font beaucoup de bruit.


Il n’y a pas beaucoup de pièces.


Je ne peux pas aller en ville à pied.


Le jardin n’est pas bien entretenu.


Je voudrais que ma maison


J’aimerais


Je préférerais


Je souhaiterais


J’adorerais


\subsection{Constituer des phrases avec les mots suivants.}
\begin{enumerate}
	\item Je - ne pas vouloir - se coucher tôt.
	\item Mon frère - être triste - sa copine - partir - en vacances.
	\item Ils - avoir - peur - pleuvoir -.
	\item Elle - souhaiter - que tu - faire - la vaisselle..
	\item Nous - vouloir - aller à ce concert.
\end{enumerate}
\subsection{Souhaits pour cette année}
\begin{enumerate}
\item Je voudrais que mes parents

\item Je voudrais que mes parents

\item Je voudrais

\item Je voudrais

\end{enumerate}

\end{document}
