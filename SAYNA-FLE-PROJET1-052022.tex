\documentclass[12pt]{article}
\usepackage[margin=10mm]{geometry}
\usepackage[T1]{fontenc}
\usepackage[utf8]{inputenc}
\usepackage[french]{babel}
\usepackage{array}
\usepackage{hyperref}
\usepackage{pifont}
\usepackage{colortbl}
\usepackage{color}
\newcommand{\und}[1]{\underline{#1}}

\author{Jean Lucien RANDRIANANTENANA}
\title{Vers un français
impeccable (B1)\\SAYNA-FLE-PROJET1-052022}
\begin{document}
\maketitle
\tableofcontents
\newpage
\section{Les temps du passés}
\subsection{Je complète le texte en conjuguant le verbe entre parenthèse à l'imparfait ou au passé composé}
Le mois dernier, je \und{me suis rendu} à la foire aux vins de Strasbourg. C\und{' était} un grand événement
alors j\und{' ai préférer} prendre les transports en commun pour y aller. Il y \und{avait} une foule
innombrable. Ce salon  \und{a présenté} le moyen de développer mon réseau puisque je \und{souhaitait} faire carrière dans une maison vinicole.
Pour valider mon diplôme, j\und{' ai dû} trouver un stage. Je \und{suis allé} visiter différents stands et j
\und{'ai rencontré} de nombreux professionnels. Je \und{'ai eu} un bon contact avec un des responsables.
Il m'\und{a posé} beaucoup de questions et à la fin de l’entretien, il \und{a voulu} que je lui donne mon
CV. Une semaine plus tard, il m'\und{a appelé} pour passer un entretien et il m'\und{a embaucher}.
\subsection{Ajouter deux phrases au plus-que-parfait dans le texte précédent}
\begin{enumerate}
	\item
	\item
\end{enumerate}

\subsection{Compléter le texte en conjuguant le verbe à l’imparfait ou au passé composé}

C'\und{était} un jour de février. Il \und{fesait} une température glaciale et il \und{a plu} des cordes.
Je \und{revenais} du travail quand j'\und{ai rencontré} une ancienne collègue. Elle \und{ne m'a pas vu} immédiatement.
Je \und{l'avais appelé}, je \und{me suis présenté} et nous \und{avons discuté} à l’abri dans la rue.
Elle \und{m'a expliqué} qu’elle \und{avait} divorcée et qu’elle \und{a eu} des enfants. Nous \und{nous somme souvenu} des anecdotes du travail quand j'\und{ai réalisé} qu’il \und{était} déjà tard.
Ma femme m'\und{a attendu} pour le repas. Alors, j\und{'ai prit} son numéro de téléphone et nous \und{nous somme salué}.
\subsection{Ajouter quatre phrases au plus-que-parfait dans le texte précédent}
\begin{enumerate}
	\item
	\item
	\item
	\item
\end{enumerate}


\section{Indicatif ou subjonctif}
\subsection{Conjuguer le verbe entre parenthèses au subjonctif présent.}
\begin{enumerate}
	\item J’ai bien peur qu’elle ne \und{puisse} pas venir à mon anniversaire.
	\item Mon père est si content que je \und{fesse} des études.
	\item Nous économisons pour que notre projet de tour du monde \und{se réalise}.
	\item De crainte que les enfants ne les \und{découvrent}, les parents cachent les cadeaux.
	\item J’aimerais beaucoup que tu \und{vienne} avec moi voir cette exposition.
	\item Il est important que tu \und{sache} la vérité.
	\item Je ne pense pas qu’elle \und{soi} de très bonne humeur aujourd'hui.
	\item Ils sont ravis que leur fils \und{ait} un travail.
	\item Ils voudraient que nous y \und{allions} avec eux.
	\item Je suis tellement contente que tu \und{veuilles} bien m’aider.
\end{enumerate}

\subsection{Choisir la forme correcte du verbe.}
\begin{enumerate}
	\item J’espère vraiment que tu obtiennes.
	\item Nous aimerions que nous partions en vacances.
	\item Elle a fait beaucoup d’effort pour que nous passions une bonne soirée.
	\item Je crois qu’il ne pourra pas venir.
	\item Je ne trouve pas que cette robe lui va bien.
	\item C’est le livre que tu connais.
	\item Je ne veux pas que tu sois déçue.
	\item Ils ont eu peur que notre train parte avec du retard.
	\item Je doute que le chauffeur de taxi sache comment aller chez elle.
	\item Il est certain que ses notes sont à la hauteur de son investissement personnel.
\end{enumerate}
\subsection{Répondre aux questions suivantes}
\begin{enumerate}
	\item Tu crois qu’ils sont déçus ? \\
	      \ding{50} Non, je ne crois pas qu'il soient déçues.
	\item Vous pensez qu’elle a assez d’argent ? \\
	      \ding{50} Non, nous ne pensons pas qu'elle a assez d'argent.
	\item Tu trouves que ce plat est bon ? \\
	      \ding{50} Non, je ne trouve pas que ce plat est bon.
	\item Tu ne crois pas qu’elle soit malade ? \\
	      \ding{50} Si, je crois qu'elle est malade.
	\item Vous trouvez qu’il fait froid ? \\
	      \ding{50} Oui, je trouve qu'il fasse froid.
\end{enumerate}
\subsection{Transformation de texte}
\begin{enumerate}
	\item Ma maison est petite et un peu sombre.\\
	      \ding{50} Je voudrais que ma maison soit petite et un peut sombre.
	\item Les voisins font beaucoup de bruit.\\
	      \ding{50} J’aimerais que les voisin fassent beaucoup de bruit.
	\item Il n’y a pas beaucoup de pièces.\\
	      \ding{50}  Je préférerais qu'il n'y ait pas beaucoup de pièces
	\item Je ne peux pas aller en ville à pied.\\
	      \ding{50}  Je souhaiterais que je n'aille pas en ville à pied.
	\item Le jardin n’est pas bien entretenu.\\
	      \ding{50}  J’adorerais que le jardin soit bien entretenu.
\end{enumerate}
\subsection{Constituer des phrases avec les mots suivants.}
\begin{enumerate}
	\item Je - ne pas vouloir - se coucher tôt.\\
	      \ding{50} Je ne veut pas me coucher tôt.
	\item Mon frère - être triste - sa copine - partir - en vacances.\\
	      \ding{50} Mon frère est triste, sa copine est part en vacance.
	\item Ils - avoir - peur - pleuvoir -.\\
	      \ding{50} Ils ont peur qu'il pleut
	\item Elle - souhaiter - que tu - faire - la vaisselle.\\
	      \ding{50} Elle souhaite que tu fasse la vaisselle.
	\item Nous - vouloir - aller à ce concert.\\
	      \ding{50} Nous voulions aller à ce concert.
\end{enumerate}
\subsection{Souhaits pour cette année}
\begin{enumerate}
	\item Je voudrais que mes parents soient en bonne santé.
	\item Je voudrais que mes parents voient réussir mes études
	\item Je voudrais partir en vacance pendant le mois de décembre.
	\item Je voudrais renforcé mes capacités à écrire et à parler en français.
\end{enumerate}

\newpage
\section{L’accord du participe-passé des verbes pronominaux}
\textbf{\und{Consigne :}} Sélectionner la bonne forme Méfiez-vous des verbes à la forme pronominale. N'accordez le participe passé avec le sujet que si ME = MOI, TE = TOI, etc. ; jamais si ME = A MOI, TE = A TOI, etc.
\vfill
\begin{center}
\begin{tabular}{|>{\centering\arraybackslash}m{3.5cm}|>{\centering\arraybackslash}m{3.5cm}|>{\centering\arraybackslash}m{3.5cm}|>{\centering\arraybackslash}m{3.5cm}|}
	\hline
	\multicolumn{4}{|c|}{Elles se sont (décidé)} \\
	\hline
	decidé & decidée & decidés & \cellcolor[gray]{0.9} \und{decidées}        \\
	\hline
\end{tabular}\vfill

\begin{tabular}{|>{\centering\arraybackslash}m{3.5cm}|>{\centering\arraybackslash}m{3.5cm}|>{\centering\arraybackslash}m{3.5cm}|>{\centering\arraybackslash}m{3.5cm}|}
	\hline
	\multicolumn{4}{|c|}{Ils se sont (répondu)} \\
	\hline
	répondu & répondue &\cellcolor[gray]{0.9} \und{répondus} & répondues \\
	\hline
\end{tabular}\vfill

\begin{tabular}{|>{\centering\arraybackslash}m{3.5cm}|>{\centering\arraybackslash}m{3.5cm}|>{\centering\arraybackslash}m{3.5cm}|>{\centering\arraybackslash}m{3.5cm}|}
	\hline
	\multicolumn{4}{|c|}{Nous nous étions (écrit)} \\
	\hline
	écrit & écrite & \cellcolor[gray]{0.9} \und{écrits} & écrites              \\
	\hline
\end{tabular}\vfill

\begin{tabular}{|>{\centering\arraybackslash}m{3.5cm}|>{\centering\arraybackslash}m{3.5cm}|>{\centering\arraybackslash}m{3.5cm}|>{\centering\arraybackslash}m{3.5cm}|}
	\hline
	\multicolumn{4}{|c|}{Elle s'est (renseigné)}      \\
	\hline
	renseigné & \cellcolor[gray]{0.9} \und{renseignée} & renseignés & renseignées \\
	\hline
\end{tabular}\vfill

\begin{tabular}{|>{\centering\arraybackslash}m{3.5cm}|>{\centering\arraybackslash}m{3.5cm}|>{\centering\arraybackslash}m{3.5cm}|>{\centering\arraybackslash}m{3.5cm}|}
	\hline
	\multicolumn{4}{|c|}{    Elle s'est (essayé) à ce jeu} \\
	\hline
	 \cellcolor[gray]{0.9} \und{essayé} & essayée& essayés& essayées\\
	\hline
\end{tabular}\vfill

\begin{tabular}{|>{\centering\arraybackslash}m{3.5cm}|>{\centering\arraybackslash}m{3.5cm}|>{\centering\arraybackslash}m{3.5cm}|>{\centering\arraybackslash}m{3.5cm}|}
	\hline
	\multicolumn{4}{|c|}{ Ils se sont (rendu) à la gare} \\
	\hline
	rendu & rendue & \cellcolor[gray]{0.9} \und{rendus} & rendues                    \\
	\hline
\end{tabular}\vfill

\begin{tabular}{|>{\centering\arraybackslash}m{3.5cm}|>{\centering\arraybackslash}m{3.5cm}|>{\centering\arraybackslash}m{3.5cm}|>{\centering\arraybackslash}m{3.5cm}|}
	\hline
	\multicolumn{4}{|c|}{Elles ne se sont pas (nui)} \\
	\hline
	 nui & nuie & nuis & \cellcolor[gray]{0.9} \und{nuies}  \\
	\hline
\end{tabular}\vfill

\begin{tabular}{|>{\centering\arraybackslash}m{3.5cm}|>{\centering\arraybackslash}m{3.5cm}|>{\centering\arraybackslash}m{3.5cm}|>{\centering\arraybackslash}m{3.5cm}|}
	\hline
	\multicolumn{4}{|c|}{Elle dit : "je me suis (offert) une croisière"} \\
	\hline
	offert & offerte & offerts & \cellcolor[gray]{0.9} \und{offertes}                                \\
	\hline
\end{tabular}\vfill

\begin{tabular}{|>{\centering\arraybackslash}m{3.5cm}|>{\centering\arraybackslash}m{3.5cm}|>{\centering\arraybackslash}m{3.5cm}|>{\centering\arraybackslash}m{3.5cm}|}
	\hline
	\multicolumn{4}{|c|}{ Nous nous sommes (efforcé) d'aider nos maris} \\
	\hline
	efforcé & efforcée & efforcés & \cellcolor[gray]{0.9} \und{efforcées}                           \\
	\hline
\end{tabular}\vfill

\begin{tabular}{|>{\centering\arraybackslash}m{3.5cm}|>{\centering\arraybackslash}m{3.5cm}|>{\centering\arraybackslash}m{3.5cm}|>{\centering\arraybackslash}m{3.5cm}|}
	\hline
	\multicolumn{4}{|c|}{ Tu t'étais (ému) des reproches de ton mari} \\
	\hline
	ému & émue & émus & \cellcolor[gray]{0.9} \und{émues}                                         \\
	\hline
\end{tabular}\vfill
\end{center}


\section{Discours indirect}
 \subsection{Transformer les phrases au discours indirect}
 \begin{enumerate}
\item « Tu as fini de faire ta valise ? » \\
	\ding{50} Elle me demande... 
 \item  « Mange tes légumes ! » \\
 \ding{50}
 \item  « Elle préfère quoi ? » \\
 \ding{50}
 \item  « Oh non, c’est moche ! » \\
 \ding{50}
 \item  « Vous arrivez quand ? » \\
 \ding{50}
 \item  « Ne laisse pas la lumière ! » \\
 \ding{50}
 \item  « Je m’appelle Laure. » \\
 \ding{50}
 \item  « Tu as fait quoi hier ? » \\
 \ding{50}
 \item  « Qu’est-ce qui t’intéresse ? » \\
 \ding{50}
 \item  « On va au restaurant ? »\\
 \ding{50}
\end{enumerate}
\subsection{Retranscrire ce dialogue au discours indirect }
\begin{enumerate}
	\item Mère : « Tu as fait tous tes devoirs ? » \\
	\ding{50} Elle a demander si j'ai fait tous mes devoir.
	\item Fils : « Non, mais, je vais les faire. » \\
	\ding{50} Il a dit, non ,mais qu'il allait les faire
	\item Mère : « Quoi ?!?! Mais, qu’est-ce que tu as fait pendant mon absence ? » \\
	\ding{50}
	\item Fils : « J’ai téléphoné à un copain pour un projet de l’école. » \\
	\ding{50}
	\item Mère : « Ne me raconte pas d’histoires ! » \\
	\ding{50}
	\item Fils : « Mais si, je te promets ! Je fais l’exposé avec Xavier, tu le connais. »\\
	\ding{50}
	\item Une mère rentre du travail et demande à son fils s’il a fait tous ses devoirs…\\
	\ding{50}
\end{enumerate}


\end{document}
